\section{Results}
\subsection{Mimicking human jaw motion}
\begin{itemize}
    \item human jaw motion from motion capture
    \item results of PCA on human jaw motion
    \item show graphs of human jaw motion vs robotic jaw motion
    \item show graphs of the force during chewing for human vs robot
\end{itemize}

\subsection{Motion captured chewing trajectories}

\subsection{Position control}

\paragraph{Speed and accuracy}
To evaluate the performance of the position control system, we investigated the robot's ability to track a predefined trajectory at varying speeds. 
In this context, “speed” refers to the time interval between consecutive trajectory points. While the motion capture data was recorded at 120 Hz 
(approximately every 8.3 ms), the playback was tested at longer intervals: 40 ms (25 Hz), 50 ms (20 Hz), 60 ms (16.67 Hz), 70 ms (14.29 Hz), 80 ms 
(12.5 Hz), 90 ms (11.11 Hz), and 100 ms (10 Hz).

For this test, a 10-second segment of randomly selected chewing motion was used. We recorded both the target and actual actuator lengths across all 
actuators. Figure~\ref{fig:actuator_delays_2} illustrates the tracking performance of actuator 2 at the various speeds. To quantify timing discrepancies, we 
conducted a cross-correlation analysis between the target and actual actuator lengths, providing both time delays (Figure~\ref{fig:actuator_delays_2}) 
and cross-correlation coefficients (Figure~\ref{fig:actuator_rhos_2}).

The results show that at a sampling interval of 100 ms (10 Hz), the robot tracks the trajectory well, with an average delay of ~0.7 s and 
cross-correlation coefficients exceeding 0.975. However, performance degrades significantly at higher speeds (i.e., shorter intervals), 
as reflected in both increased delays and a drop in correlation values—especially below 80 ms, where the tracking error becomes pronounced 
and key trajectory peaks are missed.

\begin{figure}[H]
    \centering
    \includegraphics[width=\textwidth]{figures/actuator_2_trajectories.png}
    \caption{Performance of position control of actuator 2 across different time intervals between trajectory points.}
    \label{fig:position_control}
\end{figure}

\begin{figure}[H]
    \centering
    \includegraphics[width=0.9\textwidth]{figures/actuator_delays.png}
    \caption{Delays between target and actual actuator length across different time intervals.}
    \label{fig:actuator_delays_2}
\end{figure}

\begin{figure}[H]
    \centering
    \includegraphics[width=0.9\textwidth]{figures/actuator_rhos.png}
    \caption{Cross-correlation coefficients across actuators for different time intervals.}
    \label{fig:actuator_rhos_2}
\end{figure}

\subsection{Force analysis}

\subsubsection{Maximum force}

The robot's vertical force generation capabilities were assessed using manual control mode normally used to set the robot's home position during calibration. 
The platform was driven at its highest height under active force feedback while avoiding structural failure. The upper bound of the robot's vertical force 
output was constrained by the stiffness of the upper jaw structure, which visibly bent under high vertical force. table~\ref{tab:max_force} shows the 
maximum forces recorded by the three load cells during this test, see Figure %TODO
for the load cell positions. The results shows that most of the vertical force is applied on the back load cells, which reflects the 
anatomical load pattern during full occlusion, where the molars bear the majority of chewing forces. The total vertical force output of the robot is 
315.98 N, which is well within the average occlusal force during chewing, although below the maximum bit force from Table~\ref{tab:functional_criteria}.
%TODO: find average bite force during chewing paper
\begin{table}[H]
    \centering
    \begin{tabular}{p{4cm} p{2cm} p{2cm} p{2cm}}
        \toprule
        \textbf{Load Cell} & \textbf{$F_{z,max}$ (N)} & \textbf{$F_{y,max}$ (N)} & \textbf{$F_{x,max}$ (N)} \\
        \midrule
        Back Right Load Cell & 124.63 & x & x \\
        Back Left Load Cell & 124.63 & x & x  \\
        Front Load Cell & 66.28 & x & x  \\
        \midrule
        \textbf{Total Force} & \textbf{315.98} & x & x \\
        \bottomrule       
    \end{tabular}
    \caption{Maximal forces recorded by the load cells during the force test.}
    \label{tab:max_force}
\end{table}

%TODO: shear and protrusion forces

\subsubsection{Force feedback distribution}

To evaluate the spatial resolution of the force sensing system, we applied a simple vertical trajectory while placing a piece of gum-like material 
at three distinct positions between the teeth: back right, front, and back left. The vertical force output from each of the three load cells was 
recorded (Figure~\ref{fig:force_distribution_gum}).

The plot clearly shows three distinct peaks corresponding to the three test positions, confirming the system's capability to localize force application 
across the dental arch. Additionally, when force is applied at the back, the front load cell registers a smaller secondary response. This is consistent 
with the mechanical coupling in the mounting structure, as the front load cell is physically situated between the two rear ones.

\begin{figure}[H]
    \centering 
    \includegraphics[width=0.8\textwidth]{figures/ForceDistributionGum.png}
    \caption{Vertical force output across three load cells during chewing with localized gum placement.  
    The red lines separate the three gum positions: back right, front, and back left.}
    \label{fig:force_distribution_gum}
\end{figure}

