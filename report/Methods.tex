\section{Methods}

\subsection{Required workspace and forces --> criteria for design}
\begin{itemize}
    \item max chewing force across literature
    \item max range of motion across literature
\end{itemize}

\subsection{Mechanical design}
\begin{itemize}
    \item goal is to create a robotic jaw that can mimic the motion and force of human chewing 
    \item 6dof stewart platform to be able to mimic the motion of the jaw
    \item linear actuators instead of rotary servo motors to have more efficient force transmission + simpler kinematics + more rigid structure
    \item choice of actuators based on the required force to mimic human chewing (speed less important as jaw can chew even if slow) + required length to reach the desired range of motion + feedback to control in position
    \item choosing the dimensions of the stewart platform based on the size of the actuators + working space of the robot
    \item choice of structure/material to hold upper jaw to be rigid enough to not deform under the forces applied by the actuators 
    \item 3 axis load cells to measure the force applied by the jaw
    \item so far 3d printed teeth/jaw but to be changed in the future
\end{itemize}

\subsection{Control}
\begin{itemize}
    \item electronics schematics ?
    \item inverse kinematics
    \item finding intial position of the actuators
    \item PID control for position
    \item state machine that will help for later coordination with other modules like tongue/saliva
    \item gui for user friendly use ?
\end{itemize}

\subsection{human jaw motion with motion capture}
\begin{itemize}
    \item choice of placement of markers based on other papers
    \item describing the experiments we desired
    \item PCA for synergy control ?
\end{itemize}




