\section{Conclusion}

This project presented the design and development of a novel chewing robot capable of reproducing basic jaw motions using a Stewart platform. 
The mechanical architecture, control system, and data processing pipeline were developed with extensibility and modularity in mind, laying the 
groundwork for a more complete biomimetic mastication system.

The robot successfully replicates simplified human jaw trajectories using motion capture data and demonstrates the ability to apply significant 
occlusal forces. The modular control system and mechanical design allow for future integration of additional features, such as a tongue, saliva 
module, or esophagus. The inclusion of force sensing also enables basic safety mechanisms, and the current hardware supports a wide vertical 
range of motion suitable for chewing tasks.

Despite these achievements, the robot in its current form is not yet capable of full chewing functionality. Limitations in mechanical compliance, 
control accuracy, sensing precision, and tooth design prevent accurate and robust reproduction of natural mastication. Additionally, the current 
motion capture methodology introduces errors that limit the fidelity of recorded trajectories.

Nonetheless, the system serves as a strong foundation for future work. Improvements in control strategy, sensing, compliance, and anatomical 
fidelity—combined with more accurate motion datasets—will enable the robot to more closely replicate the biomechanics of human chewing. This 
work offers a promising starting point for research in food processing, oral biomechanics, and human-robot interaction involving mastication.
