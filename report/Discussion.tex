\section{Discussion}

\subsection{Summary of Findings}

The chewing robot satisfies every functional requirement listed in Table~\ref{tab:functional_criteria}—save for peak bite force—while still matching the average occlusal forces. 
Motion-capture trajectories are successfully replicated at one-twelfth of the human chewing rates, demonstrating that the 6-DoF platform can mimic human chewing kinematics. 
Integrated tri-axial load-cell arrays provide spatially resolved force data, enabling 3-D feedback for future closed-loop force control and basic safety mechanism. 
A layered architecture—mechanics, modular robot controller, and a user-friendly GUI—keeps the system open for upgrades such as an actuated tongue, saliva circuit, or esophagus module. 
In short, the current design delivers human-scale motion and average chewing forces today, while leaving clear hooks for future biomimetic extensions.

\subsection{Limitations}

Although this work represents an important first step towards a complete chewing robot, several limitations remain that currently prevent the system from 
accurately mimicking real mastication.

\subsubsection{Mechanical design}
The robot currently lacks a closed mouth, which would cause food to fall out during chewing. For a functional mastication process, 
particularly with soft or fragmented food items, a sealed oral cavity is essential to retain and guide the bolus.

The current teeth are 3D printed in PLA, a material that is not hard enough to reproduce the mechanical properties of real enamel. Furthermore, the surface finish of 3D printed PLA is inherently rough, which introduces friction and causes 
jitter when the teeth slide against each other. This affects both motion fidelity and safety of the robot.

Structural bending of the upper jaw during high-force tests limits vertical force output to ~300 N—well below the theoretical 1155 N—introducing instability and limiting our ability to reproduce realistic occlusal forces. 

The current Stewart platform is non-compliant and replays motion with limited precision—an unfavorable combination for full occlusion, where natural teeth 
can slide and interact with high dexterity while the robot risks damage during such contact.

\subsubsection{Electronics and control}

The current control loop lacks force feedback and bio-inspiration. It also introduces a 0.7 s delay at 10 Hz, which is too high for real-time interaction 
with future modules like the tongue or saliva systems. 

Position feedback from the built-in potentiometers is too coarse for precise motion tracking. Higher-resolution sensors, such as optical encoders, would improve accuracy. 
Additionally, relying on actuators with integrated feedback limits selection; using external sensors could allow for faster, more compact motors.

\subsubsection{Motion capture recordings}
\label{sec:motion_capture_limitations}

The accuracy of the recorded jaw motion is limited by the motion capture system, with a calibration error of around 0.3 mm—significant given that chewing displacements 
occur on a millimeter scale. Additionally, tracking was done using a marker on the gnathion, a skin-based reference point, which introduces error due to skin movement 
relative to the jawbone. As a result, the recorded trajectories lack full biomechanical accuracy. Alternative methods like kinesiography \cite{kinesiograph}, which track a magnet fixed to 
the jaw, offer higher spatial and temporal precision and would be better suited for generating reliable chewing datasets.

\subsection{Future Work}

This section outlines key directions for improving both the mechanical design and control of the chewing robot to move closer to a functional and autonomous chewing system.

\subsubsection{Mechanical design improvements}

Future designs should feature a closed oral cavity using a flexible membrane—such as latex or polyurethane— and a transparent membrane would allow visual monitoring of the food during mastication. 

The current PLA teeth should be replaced with more rigid materials, such as 3D-printed resin or anatomical dental models, to better replicate the mechanical properties of real teeth. 

The upper jaw assembly should be reinforced to allow for higher vertical force. This could be achieved by welding structural ribs onto the mounting plate or attaching 
an external aluminum rod for additional support. 

Additionally, a compliant layer between the lower jaw and the platform could be added to allow for more realistic interaction between maxillary and mandibular teeth during occlusion.

Finally, the tongue, saliva, and esophagus modules should be developed to enhance the robot's functionality and mimic the full chewing process. Two cameras, RGB and thermal, should be mounted inside the mouth to monitor it. 

\subsubsection{Control system improvements}

The current closed-loop position-based strategy should be changed to managing both position and force in a 
dynamic, compliant manner, such as impedance control \cite{impedance_control}.

Future iterations of the software could leverage the recorded human chewing motion data to identify a small set of representative motion patterns using PCA. Mixing these trajectories 
would allow the robot to switch between different chewing motions based on sensory input or task requirements. This would enable autonomous mastication and food processing, paving 
the way for biologically accurate studies of chewing mechanics.

To implement the above approach, the suitability of the current motion dataset must be assessed. If inadequate, new recordings—ideally using more accurate motion capture 
methods—will be necessary.


