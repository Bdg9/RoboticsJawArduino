\section{Discussion}

\subsection{Limitations}

Although this work represents an important first step towards a complete chewing robot, several limitations remain that currently prevent the system from 
accurately mimicking real mastication.

\subsubsection{Mechanical design}
%\paragraph{Incomplete oral enclosure.}
One major limitation is that the robot does not currently feature a closed mouth. This means that food placed between the jaws is prone to falling out during 
chewing. For a functional mastication process, particularly with soft or fragmented food items, a sealed oral cavity is essential to retain and guide the bolus.

%\paragraph{Non-biological tooth material and surface.}
The current teeth are 3D printed in PLA, a material that is neither hard nor stiff enough to reproduce the mechanical properties of real enamel. As a result, 
they deform under load and are susceptible to wear. Furthermore, the surface finish of 3D printed PLA is inherently rough, which introduces friction and causes 
jitter when the teeth slide against each other. This affects both motion fidelity and safety of the robot.

%\paragraph{Structural compliance of the upper jaw.}
During high-force tests, the upper jaw frame was observed to bend, limiting the system's maximum vertical force to around 300 N—well below the theoretical 
capacity of 1155 N. This compliance introduces uncontrolled movement and restricts our ability to reproduce realistic occlusal forces.

%\paragraph{Incompatibility of stiffness and precision.}
The current Stewart platform is both non-compliant and relatively imprecise in position tracking. This is a problematic combination for full occlusion: human teeth 
can slide and interact with high dexterity, while the robot risks damage during such contact. Introducing compliance, for instance with a soft layer under the mandibular 
teeth, could improve robustness and better replicate human chewing mechanics.

\subsubsection{Electronics and control}
%\paragraph{Limitations of the control strategy.}
The control strategy implemented in this prototype is intentionally simple to validate basic mechanical function. However, it lacks both force regulation and 
bio-inspiration. There is no feedback loop controlling the force applied to the upper jaw, and the chewing trajectory is open-loop. Additionally, the system 
introduces a 0.7 s delay at 10 Hz, which is too high for real-time coordination with future modules such as the tongue or saliva system.

%\paragraph{Inaccurate position feedback and actuator selection.}
The potentiometers used for position feedback are not sufficiently precise to capture small actuator displacements. Optical encoders or alternative sensing methods 
would offer improved resolution. Moreover, since we currently rely on motors with built-in position feedback, this constrains actuator selection. Exploring motors 
with similar force capabilities but higher speed and no integrated feedback could enhance performance.

\subsubsection{Motion capture recordings}
\label{sec:motion_capture_limitations}

%\paragraph{Optitrack precision.}
The quality of the recorded human jaw motion is limited by the precision and methodology of the motion capture system. The calibration error of approximately $0.3$ mm is 
relatively high in the context of mastication, where displacements typically occur on the scale of millimeters. This level of uncertainty can significantly 
affect the fidelity of the extracted trajectories.

%\paragraph{Skin motion.}
Moreover, the motion was tracked using a reflective marker placed on the gnathion, a point located on the skin. Since the skin exhibits independent movement relative to 
the underlying jaw bone during chewing, the recorded trajectory does not accurately reflect true mandibular motion. This inherently reduces the biomechanical validity of the data.

Alternative approaches such as kinesiography, which involves tracking a small magnet attached to the jaw, offer better temporal and spatial resolution by directly capturing 
bone motion. These methods would be more suitable for building reliable and reusable datasets of human chewing trajectories.

\subsection{Future Work}

This section outlines key directions for improving both the mechanical design and control architecture of the chewing robot to move closer to a functional and autonomous chewing system.

\subsubsection{Mechanical design improvements}

%\paragraph{Closing the mouth and food retention.}  
A major limitation of the current prototype is the open oral cavity. Sealing the mouth using a flexible membrane—such as latex or polyurethane—would prevent food from falling out 
during chewing and improve user safety. A transparent membrane would additionally allow visual monitoring of the food during mastication.

%\paragraph{Tongue, saliva, and esophagus modules.}  
The mechanical design already accommodates future modules including a tongue and esophagus. The tongue would help reposition the food between cycles or push it into the esophagus. 
Integration of a saliva system would enable studies on bolus formation and realistic food breakdown. To monitor these internal processes, a small camera will be mounted inside the mouth.

%\paragraph{Tooth design improvements.}  
The current PLA teeth are not mechanically suitable for realistic chewing. Future iterations should use more rigid and smoother materials, such as 3D-printed resin or 
anatomical dental models, to reduce deformation, surface friction, and jitter.

%\paragraph{Upper jaw rigidity.}  
To address observed bending under load, the upper jaw assembly should be reinforced. This could be achieved by welding structural ribs onto the mounting plate or attaching 
an external aluminum rod for additional support.

%\paragraph{Weight reduction.}  
%The use of heavy steel components contributes to the robot's high mass. Future iterations could explore alternative materials to reduce weight while preserving structural integrity.

%\paragraph{Compliance under mandibular teeth.}  
Introducing a compliant layer beneath the mandibular teeth would allow controlled deformation during occlusion. This would reduce risk of damage and improve fidelity 
in reproducing natural sliding contact between teeth.

\subsubsection{Control system improvements}

%\paragraph{Force-aware.}  
The current closed-loop position-based strategy is insufficient for coordinated mastication. Future work should explore alternative control methods managing both position and force in a 
dynamic, compliant manner, such as impedance control. 
\textbf{TODO: cite relevant impedance control paper}

%\paragraph{Bio-inspired control, towards autonomous chewing.}  
Future iterations of the software could leverage the recorded human chewing motion data to identify a small set of representative motion patterns using PCA. Mixing these trajectories 
would allow the robot to switch between different chewing motions based on sensory input or task requirements. This would enable autonomous mastication and food processing, paving 
the way for biologically accurate studies of chewing mechanics.

%\paragraph{Dataset validation and expansion.}  
To implement the above approach, the suitability of the current motion dataset must be assessed. If inadequate, new recordings—ideally using more accurate motion capture 
methods—will be necessary.


