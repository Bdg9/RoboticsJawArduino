\section{Introduction}

\begin{itemize}
    \item Jaw is one of the most complex articulation in the human body with its 6 degrees of freedom and complex joint movement
    \item Chewing robot applications: food industry, medical field, dental field, etc.
    \item Chewing robot can be used to test food texture, dental implants, orthodontic devices, etc.
    \item Chewing robot can be used to study chewing disorders and develop treatments / also understand more about the human chewing process/development throughout life
    \item State of the art chewing robots
    \item Most of them are not able to mimic the human chewing process because not 6 dof / no saliva / no tongue / not able to apply the same forces as human
    \item Our aim is to create a chewing robot that can mimic the human chewing process as closely as possible
\end{itemize}


%1. **Goal: What do you want to achieve? (e.g. KPI/New capabilities)**
%2. **Problem: What are the obstacles to achieve the goal? What is currently not possible? (Includes literature review paragraph)**
%3. **Hypothesis: How do plan to overcome the obstacles and how do you expect this to work? This is usually theoretical/mathematical models/methods**
%4. **Proof and contributions Does your hypothesis work? To what extent quantitatively?**
%5. **Conclusion: What did we learn?**