\section{Introduction}

%1. **Goal: What do you want to achieve? (e.g. KPI/New capabilities)**
%2. **Problem: What are the obstacles to achieve the goal? What is currently not possible? (Includes literature review paragraph)**
%3. **Hypothesis: How do plan to overcome the obstacles and how do you expect this to work? This is usually theoretical/mathematical models/methods**
%4. **Proof and contributions Does your hypothesis work? To what extent quantitatively?**
%5. **Conclusion: What did we learn?**

Human chewing is a complex process, involving different systems such as the jaw, teeth, tongue and saliva, all coordinated to break down food into 
a bolus that can be swallowed and digested. Chewing robots are a great tool to study this process that is not yet fully understood as they give us the opportunity 
for a closely controlled environment where each parameter can be adjusted and measured. This makes them valuable not only for advancing our understanding of 
chewing mechanics and related disorders, but also for a wide range of applications. In dentistry, they are used to test how implants and other dental devices
 wear over time. In food science, they help assess texture and flavor release during mastication. They also offer a reliable platform for studying the release of 
 active compounds in chewable medications such as medical chewing gum.\\
Nowadays, many mastication robots exist, and while most are limited in their ability to fully mimic human chewing due to restricted degrees of freedom, 
some have already made significant progress. For example, the Bristol Dento-Munch Robo-Simulator \cite{BristolChewingRobot} features 6 degrees of freedom 
(DoF), closed-loop control, force feedback, and a full set of teeth capable of replicating human chewing forces. Similarly, the robot developed by Seung-Ju 
Lee \cite{ChewingRobotLinearActuator} offers the same capabilities, with a design that more closely follows human biomechanics. Another system, by Alemzadeh 
et al. \cite{ChewingRobotGums}, includes a closed mouth and artificial saliva—two important components of realistic mastication—even though it has limited 
sensory feedback. However, none of these systems combine all critical elements: 6 DoF, position and multidirectional force feedback, a closed mouth, and 
saliva. In addition, none of them include a tongue, which plays a crucial role in directing the food towards the molars and mixing it with saliva during 
chewing.\\
This project aims at developing a chewing robot that incorporates all these features, with the goal of replicating the full chewing process as closely as 
possible. As this is the first iteration, the focus is primarily on the mechanical design and the development of a simple, closed-loop, modular, and scalable 
control system that can be extended in future versions. The robot includes a 6-DoF Stewart platform capable of replicating chewing forces, along with 
tri-axial force sensing and position feedback to monitor jaw dynamics in detail. In parallel, we created a first dataset of human chewing motion using 
motion capture, as no publicly available datasets were found. This dataset will serve as a reference for identifying chewing patterns and 
improving control strategies. By laying down a flexible and expandable foundation, this work provides a platform for future integration of key components 
such as artificial saliva flow, an artificial tongue, and adaptive neuromuscular control strategies.
